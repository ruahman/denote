\documentclass[18px]{article}
\usepackage{enumerate}

\begin{document}

\begin{enumerate}
\item cat
\item bat
\item fat
	\begin{enumerate}
	\item cat
	\item bat
	\item fat
		\begin{enumerate}
		\item cat
		\item bat
		\item fat
		\end{enumerate}
	\end{enumerate}
\end{enumerate}

\begin{itemize}
\item cat
\item bat
\item fat
\end{itemize}


\begin{enumerate}[A.]
\item cat
\item bat
\item fat
\end{enumerate}



\end{document}