% Created 2023-01-12 Thu 10:27
% Intended LaTeX compiler: pdflatex
\documentclass[11pt]{article}
\usepackage[utf8]{inputenc}
\usepackage[T1]{fontenc}
\usepackage{graphicx}
\usepackage{longtable}
\usepackage{wrapfig}
\usepackage{rotating}
\usepackage[normalem]{ulem}
\usepackage{amsmath}
\usepackage{amssymb}
\usepackage{capt-of}
\usepackage{hyperref}
\author{Diego Vila}
\date{\textit{[2023-01-10 Tue 07:24]}}
\title{finite fields}
\hypersetup{
 pdfauthor={Diego Vila},
 pdftitle={finite fields},
 pdfkeywords={},
 pdfsubject={},
 pdfcreator={Emacs 28.2 (Org mode 9.5.5)}, 
 pdflang={English}}
\begin{document}

\maketitle
\tableofcontents


\section{finite field definition}
\label{sec:org78192f1}

a finite field is a finite set of number and two operations,(+,-) that satisfy the following

\begin{enumerate}
\item if a and b are in the set then, a + b and a * b are also in the set.
\begin{itemize}
\item this is called the \uline{closed} property
\item the results always stay inside the set
\end{itemize}

\item zero exists and has the propert a + 0 = a
\begin{itemize}
\item this is called the \uline{additive identity} property
\end{itemize}

\item 1 exitst and a * 1 = a
\begin{itemize}
\item this is called the \uline{multiplicative identity} property
\end{itemize}

\item if a is in the set and -a is in the set, then a - a = 0
\begin{itemize}
\item this is called the \uline{additive inverse} property
\end{itemize}

\item if a is in the set and is not 0, then \(a^{-1}\) is in the set that make \(a \cdot a^{-1} = 1\)
\begin{itemize}
\item this is called the \uline{multiplication inverse} property
\end{itemize}

\item the \uline{order} of the set size of numbers

\item in order to acceive the following properties we need to redefine additon and multiplication for the set.
\end{enumerate}

\section{defining finite sets}
\label{sec:org63cbff5}

if the order of the set is defined as p, then the elements of the set are \(\{0,1,2,\dots,p-1\}\)

\begin{itemize}
\item these are called elements of the set
\item notice that p is alway one more than its largist element
\item p is alway prime
\end{itemize}

\begin{align*}
F_p  &= \{0,1,2,\dots,p-1\}\\
F_{11} &= \{0,1,2,3,4,5,6,7,8,9,10\}
\end{align*}

\section{finite field in python}
\label{sec:org9260966}

\begin{verbatim}
print("hello world again")
\end{verbatim}
\end{document}