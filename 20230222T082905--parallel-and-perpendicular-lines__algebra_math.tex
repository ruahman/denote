% Created 2023-02-22 Wed 09:28
% Intended LaTeX compiler: pdflatex
\documentclass[11pt]{article}
\usepackage[latin1]{inputenc}
\usepackage[T1]{fontenc}
\usepackage{graphicx}
\usepackage{longtable}
\usepackage{wrapfig}
\usepackage{rotating}
\usepackage[normalem]{ulem}
\usepackage{amsmath}
\usepackage{amssymb}
\usepackage{capt-of}
\usepackage{hyperref}
\usepackage{pgfplots}
\author{Diego Vila}
\date{\textit{[2023-02-22 Wed 08:29]}}
\title{Parallel and Perpendicular lines}
\hypersetup{
 pdfauthor={Diego Vila},
 pdftitle={Parallel and Perpendicular lines},
 pdfkeywords={},
 pdfsubject={},
 pdfcreator={Emacs 28.2 (Org mode 9.6.1)}, 
 pdflang={English}}
\begin{document}

\maketitle
\tableofcontents


\section{graph}
\label{sec:org403fdd3}
\begin{tikzpicture}
  \begin{axis}[xmin=0,xmax=20,ymin=0,ymax=20,axis lines=left,xlabel=x,ylabel=y,title={diego's graph}]
    \addplot[
      color=blue,
      domain=0:20
    ]{0.5*x+2};
    \addplot[
      color=red,
      domain=0:20
    ]{0.5*x+5};
    \addplot[
      color=green,
      domain=0:20
    ]{-2*x+20};

  \end{axis}
\end{tikzpicture}

\begin{itemize}
\item lines are parallel if they have the same slop
\item lines are perpendicular if the slop is the oposite recipical
\begin{itemize}
\item change the sign flip the fraction
\end{itemize}
\end{itemize}
\end{document}