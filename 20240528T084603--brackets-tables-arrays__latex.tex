\documentclass[18px]{article}
\usepackage{amsfonts,amssymb,amsmath}
\pagestyle{empty}
\parindent 0px

\begin{document}
The distributive property states that $a(b+c)=ab+ac$ for all $a,b,c \in \mathbb{R}$\\[6pt]
The equivalence class of $a$ is $[a]$.\\[6pt]
The set $A$ is defined to be $\{1,2,3\}$.\\[6pt]
The movie ticket cost $\$12.00$

$$2\left(\frac{1}{\sqrt{x + 1}}\right)$$
$$2\left[\frac{1}{\sqrt{x + 1}}\right]$$
$$2\left\{\frac{1}{\sqrt{x + 1}}\right\}$$
$$2\left|\frac{1}{\sqrt{x + 1}}\right|$$
$$\left.\frac{dy}{dx}\right|_{x=1}$$
$$\left(\frac{1}{1 + \left(\frac{1}{1+x}\right)} \right)$$

Tables:\\

\begin{tabular}{|c||c|c|c|c|c|}
\hline
x & 1 & 2 & 3 & 4 & 5 \\ \hline
f(x) & 11 & 12 & 13 & 14 & 15 \\ \hline 

\end{tabular}

Arrays:
\begin{align}
5x^2-9=x+3\\
5x^2-12=x
\end{align}

% this takes out numbering
\begin{align*}
5x^2-9 &=x+3\\
5x^2-12 &=x
\end{align*}

% start numbering again
\begin{align}
5x^2-9=x+3\\
5x^2-12=x
\end{align}


\end{document}