% Created 2023-01-26 Thu 10:18
% Intended LaTeX compiler: pdflatex
\documentclass[11pt]{article}
\usepackage[utf8]{inputenc}
\usepackage[T1]{fontenc}
\usepackage{graphicx}
\usepackage{longtable}
\usepackage{wrapfig}
\usepackage{rotating}
\usepackage[normalem]{ulem}
\usepackage{amsmath}
\usepackage{amssymb}
\usepackage{capt-of}
\usepackage{hyperref}
\author{Diego Vila}
\date{\textit{[2023-01-20 Fri 10:08]}}
\title{division}
\hypersetup{
 pdfauthor={Diego Vila},
 pdftitle={division},
 pdfkeywords={},
 pdfsubject={},
 pdfcreator={Emacs 27.1 (Org mode 9.6)}, 
 pdflang={English}}
\begin{document}

\maketitle
\tableofcontents

In \(F_{19}\), we know that

\begin{align*}
3 \cdot_f 7 &= 21 \mod 19 = 2\\
9 \cdot_f 5 &= 45 \mod 19 = 7\\
\end{align*}

then

\begin{align*}
2 \div_f 7 &= 3\\
7 \div_f 5 &= 9\\
\end{align*}

but how do we figure out the division we don't now 3 or 9? \(2 \div_f 7 = x\) or \(7 \div_f 5 = x\)

\section{Fermat's Little Theorem}
\label{sec:org0f6f3b1}
\begin{equation*}
n^{(p-1)}\mod p = 1
\end{equation*}
\end{document}