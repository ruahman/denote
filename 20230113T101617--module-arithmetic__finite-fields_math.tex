% Created 2023-01-17 Tue 10:26
% Intended LaTeX compiler: pdflatex
\documentclass[11pt]{article}
\usepackage[utf8]{inputenc}
\usepackage[T1]{fontenc}
\usepackage{graphicx}
\usepackage{longtable}
\usepackage{wrapfig}
\usepackage{rotating}
\usepackage[normalem]{ulem}
\usepackage{amsmath}
\usepackage{amssymb}
\usepackage{capt-of}
\usepackage{hyperref}
\author{Diego Vila}
\date{\textit{[2023-01-13 Fri 10:16]}}
\title{module arithmetic}
\hypersetup{
 pdfauthor={Diego Vila},
 pdftitle={module arithmetic},
 pdfkeywords={},
 pdfsubject={},
 pdfcreator={Emacs 28.2 (Org mode 9.5.5)}, 
 pdflang={English}}
\begin{document}

\maketitle
\tableofcontents


\section{module arithmetic}
\label{sec:org50501ef}
one of the ways we can make finite fields closed under addition, subtraction, multiplication and division
is by module arithmetic

we can define addition on a finite set using module arithmetic

7 \% 3 = 1

\begin{itemize}
\item the module is the remander

1747 \% 241 = 60

\item it helps to think of module arithmetic as a clock

\item ex:
what will be the time if now is 3:00 in 47 hours?

(3 + 47) \% 12 = 2

\item ex:
what was the time 16 hours ago?

(3 - 16) \% 12 = 11

\item ex:
what will be the time minutes from now?

(12 + 843) \% 60 = 15

\item we will be using module arithmetic when defining finite field arithmetic
\end{itemize}

\section{Finite Field Addition and Subtraction}
\label{sec:orgc61fd13}
\begin{itemize}
\item we must define addition in such a way that the results are still in the set
\begin{itemize}
\item we must make sure that addition in a finite set is \uline{closed}
\end{itemize}

\item let say we have a finite field of 19
\end{itemize}
\begin{equation*}
F_{19} = \{0,1,2, \dots 18\}
\end{equation*}

\begin{itemize}
\item where \(a,b \in F_{19}\), this means a and b are elements in \(F_{19}\)
\item for addition to be closed \(a +_f b \in F_{19}\),
\begin{itemize}
\item note \(+_f\) represents field addition so that you don't get confused with normal aditiion
\end{itemize}
\item we define addition using module arithmetic this way
\begin{itemize}
\item \(a +_f b = (a + b) \mod p\)
\end{itemize}
\item we can also define subtraction using module arithmetic
\begin{itemize}
\item \(a -_f b = (a - b) \mod p\)
\end{itemize}
\end{itemize}

\subsection{python implementation}
\label{sec:org1281ef3}

\begin{verbatim}
class FieldElement:

    def __init__(self, num, prime):
        if num >= prime or num < 0:
            raise ValueError(f"{num} is not in finite field {prime}")
        self.num = num
        self.prime = prime

    def __repr__(self):
        return f"FieldElement_{self.prime}({self.num})"

    def __eq__(self, other):
        if other is None:
            return False
        return self.num == other.num and self.prime == other.prime

    def __ne__(self, other):
        return not (self == other)

    def __add__(self, other):
        if self.prime != other.prime:
            raise TypeError("they dont belong to the same finite-field")
        num = (self.num + other.num) % self.prime
        return self.__class__(num, other.prime)

    def __sub__(self, other):
        if self.prime != other.prime:
            raise TypeError("they dont belong to the same finite-field")
        num = (self.num - other.num) % self.prime
        return self.__class__(num, other.prime)

a = FieldElement(7, 13)
b = FieldElement(12, 13)
c = FieldElement(6, 13)
d = FieldElement(8, 13)

print(a+b==c)
print(a-b==d)
\end{verbatim}

\section{Finite Field Multiplicaton and exponent}
\label{sec:org7ce4f71}
\begin{align*}
5 \cdot 3 &= 5 + 5 + 5 = 15\\
8 \cdot 7 &= 8 + 8 + 8 + \dots = 136 
\end{align*}

\begin{itemize}
\item given \(F_{19}\) we can do the same with finite-field arithmetic

\begin{align*}
5 \cdot_f 3 &= 5 +_f 5 +_f 5 = 15\mod 19 = 15\\
8 \cdot_f 7 &= 8 +_f 8 +_f 8 +_f \dots = (8 \cdot 17) \mod 19 = 136 \mod 19 = 3
\end{align*}

\item exponents are simply
\begin{align*}
7^3 = 7 \cdot_f 7 \cdot_f 7 = 343 \mod 19 = 1
\end{align*}
\end{itemize}
\end{document}